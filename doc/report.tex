\documentclass[a4]{article}
\usepackage[brazilian]{babel}
\usepackage[utf8]{inputenc}
\usepackage[T1]{fontenc}

\sloppy

\title{Métodos Computacionais\\ Seminário I}

\author{Giovanni Cupertino, Matthias Nunes}

\begin{document}

\maketitle

\section{Introdução}

	\textbf{CAS}, ou \textbf{C}omputer \textbf{A}lgebra \textbf{S}ystem, é um
	tipo de software usado na manipulação de fórmulas matemáticas. O objetivo
	principal de um \textbf{CAS} é automatizar manipulações algébricas chatas e
	muitas vezes difíceis. A principal diferença entre \textbf{CAS} e uma
	calculadora tradicional é a habilidade de lidar com equações de forma
	simbólica, e não numérica. Os usos específicos e capacidades desses sistemas
	varia muito de um para o outro, mas seu propósito continua o mesmo:
	manipulação de equações simbólicas.

\section{Implementação de um CAS}

	\subsection{Estruturas de Dados}

		Para um programa começar a manipular uma equação simbólica, primeiro é
		preciso guardar essa equação na memória. No centro de qualquer sistema
		algébrico computacional existe uma estrutura de dados (ou combinação de
		estruturas de dados) responsável por descrever uma equação matemática.
		Equações podem existir em diversas variáveis, conter referências para
		outras funções, e podem elas mesmas serem funções racionais. Não existe
		solução perfeita para uma equação representada em uma estrutura de
		dados. Uma representação pode ser eficiente para certas operações
		matemáticas, mas ruim para outras. Outra representação pode ser
		ineficiente em complexidade de tempo e espaço, mas fácil de programar.
		Essas negociações precisam ser consideradas quando escolher uma
		representação, não existe resposta absoluta ao problema.

	\subsection{Simplificação}

		Uma tarefa comum que qualquer sistema algébrico computacional precisa
		fazer é simplificar uma expressão. Simplicar uma expressão torna outras
		tarefas mais fáceis, especialmente comparar expressões que entraram em
		formas diferentes e verificar se são equivalentes. O sistema tem que
		saber como somar termos que devem ser somados e como somar expoentes
		quando seus multiplicandos tiverem a mesma base. Sistemas algébricos
		computacionais sempre devem ordenar todos os termos para chegar na forma
		canônica. Deve ter uma ordem consistente onde tudo é ordenado. Se
		expoentes podem ser polinomiais, como $x^{x^3 +3+x}$, esses expoentes
		devem ser ordenados. Isso pode seguir indefinidamente, com cada potência
		sendo mais uma polinomial complexa. Ordenação vai ter que acontecer em
		diferentes níveis para que seja consistente. Os expoentes mais
		predominantes devem ser ordenados primeiros para que os abaixo fiquem em
		seus respectivos lugares.

\end{document}
