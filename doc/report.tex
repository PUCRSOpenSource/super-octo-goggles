\documentclass[9pt]{IEEEtran}
\usepackage[brazilian]{babel}
\usepackage[utf8]{inputenc}
\usepackage[T1]{fontenc}

\sloppy

\title{Programação Paralela\\ Trabalho I}

\author{Giovanni Cupertino, Matthias Nunes, \IEEEmembership{Usuário pp12820}}

\begin{document}

\maketitle

\section{Introdução}

O objetivo do trabalho é desenvolver uma solução que ordene diversos vetores
utilizando o algoritmo Quick Sort.  Os vetores contém cem mil elementos e são,
no total, mil vetores que estão na ordem inversa de valores indo de noventa e
nove mil novecentos e noventa e nove na primeira posição até zero na ultima
posição.  Para a abordagem paralela do trabalho, utilizou-se a estrutura mestre
escravo utilizando biblioteca \textbf{MPI}. Para esta abordagem temos um dos
processos como mestre, que é responsável por mandar mensagens com o vetor a ser
ordenado, avisar aos escravos para que terminem suas funções (suicídio) e para
reorganizar o vetor na sua posição original na estrutura. Os outros processos
(escravos) recebem o vetor, o ordenam e devolvem ao mestre e, enquanto estão
ativos, ficam pedindo um vetor para ordenar.

%FOTOS vou te mandar depois

Em primeira análise é possível observar que o tempo de resposta para dois
processos é maior que quando executado sequencialmente, isso ocorre pelo fato de
que só temos 1 processo realizando a ordenação e o outro como mestre, havendo um
tempo extra de troca de mensagens entre os dois. Essas pequenas diferenças, como
a troca de mensagens, a reorganização dos vetores nas suas posições inicias e a
comunicação entre os 2 nós utilizados também reduzem a eficiência e criam
diferenças entre o speed-up ideal e o real. Devido a velocidade do algoritmo de
ordenação o tempo para realizar a tarefa depois de cinco processos permaneceu
bastante semelhante, já que os escravos terminam a sua tarefa antes mesmo do
mestre conseguir enviar uma nova para todos os processos que estão solicitando.
Algumas alternativas para maior utilização dos núcleos, para os casos com mais
de cinco, seriam possuir mais de um mestre permitindo maior distribuição de
vetores aos escravos(cada um controlando escravos diferentes e passando os
vetores quando lhes forem requeridos) e a utilização de um algoritmo de
ordenação mais lento que proporcionaria mais tempo ao mestre para distribuir
vetores aos escravos, ou seja, antes que tivesse novas requisições.

\end{document}
